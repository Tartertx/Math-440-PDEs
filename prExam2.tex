\documentclass[12 pt]{article}        	%sets the font to 12 pt and says this is an article (as opposed to book or other documents)
\usepackage{amsfonts, amssymb}
\usepackage{amsmath}
\usepackage{graphicx}
\usepackage{esint}
\usepackage{float}
\usepackage{mathrsfs}

\usepackage[left=2cm,right=2.5cm,top=2cm,bottom=1.5cm]{geometry}

\pagestyle{myheadings}

\newcommand{\eqn}[0]{\begin{array}{rcl}}%begin an aligned equation - allows for aligning = or inequalities.  Always use with $$ $$
\newcommand{\eqnend}[0]{\end{array} }  	%end the aligned equation

\newcommand{\qed}[0]{$\square$}

% personalized commands
\newcommand{\inv}{^{-1}}
\newcommand{\bij}{\mathrlap{\hookrightarrow}\rightarrow}
\newcommand{\onto}{\twoheadrightarrow}
\newcommand{\z}{\mathbb Z}
\newcommand{\real}{\mathbb R}
\newcommand{\q}{\mathbb Q}
\newcommand{\complex}{\mathbb C}
\newcommand{\n}{\mathbb N}
\newcommand{\cont}{\lightning}
\newcommand{\osub}{\stackrel{\circ}{\subset}}
\newcommand{\U}{\mathscr{U}}
\newcommand{\B}{\mathscr{B}}

\usepackage{amsthm} % Recommended for theorem-like environments
\usepackage[usenames,dvipsnames]{xcolor}

\newtheorem{definition}{Definition}
\newtheorem{prop}{Proposition}
\newtheorem{ex}{Example}
\newtheorem{thm}{Theorem}
\newtheorem{lem}{Lemma}

\title{Math 440 Exam 2 Practice Test}
\begin{document}
\maketitle
\begin{center}
    Timothy Tarter

    James Madison University

    Department of Mathematics
\end{center}

\section{Describe / Explain:}
\subsection{Laplace's Equation (include well-posedness, uniqueness, and the mean value proposition).}
{\color{red}      answer goes here}

\subsection{Linear Operators, provide an example of something which is a linear operator, and provide an example of something which is not a linear operator.}
{\color{red}      answer goes here}

\section{Prove Orthogonality of Sines.}
{\color{red}      answer goes here}

\section{Solve $\frac{\partial u}{\partial t} = k \frac{\partial^2 u}{\partial x^2}$ with certain BC's and IC's, addressing $\lambda >0, = 0, <0$.}
{\color{red}      answer goes here}

\section{Solve $\frac{\partial u}{\partial t} = k \frac{\partial^2 u}{\partial x^2}$ at equilibrium in polar coordinates over the interval $(-L, L)$.}
{\color{red}     
Note that the 1D heat equation can be expressed in polar coordinate with axial symmetry (meaning that $\frac{\partial^2 u}{\partial \theta^2} =0$) as follows:
\begin{equation}
    0 = \frac{1}{r}\frac{\partial}{\partial r}(r \frac{\partial u }{\partial r}).
\end{equation}
We also get the following BCs for free:
\begin{align}
    u(-L) &= u(L)\\
    \frac{\partial u }{\partial r}(-L) &= \frac{\partial u }{\partial r}(L)\\
    |u(0)|&<\infty.
\end{align}
















}

\section{Solve Laplace's Equation on a rectangle, with LHS, Top, and Bottom $=0$, RHS $= f_2(y)$.}




\section{Extra Resources Which You Gotta Know or You Might Be Cooked:}
{\color{blue} 
Trig Identities: 
\begin{align}
    sin(a+b) = sin(a)cos(b)+cos(a)sin(b)\\
    cos(a+b) = cos(a)cos(b) - sin(a)sin(b)\\
    sin(a-b) = sin(a)cos(b)-cos(a)sin(b)\\
    cos(a-b) = cos(a)cos(b) + sin(a)sin(b)
\end{align}

$\newline$
If BC's are
\begin{itemize}
    \item $u(0,t) = 0 = u(L,t)$ then 
    \begin{equation}
        u(x,t) = \sum_{n=1}^\infty A_n e^{-kt(\frac{n\pi}{L})^2}sin(\frac{n\pi x}{L})
    \end{equation}
    \item $\frac{\partial u}{\partial x}(0,t) = 0 = \frac{\partial u}{\partial x}(L,t)$ then 
    \begin{equation}
        u(x,t) = \sum_{n=1}^\infty B_n e^{-kt(\frac{n\pi}{L})^2}cos(\frac{n\pi x}{L})
    \end{equation}
\end{itemize}
where 
\begin{align}
    A_n = \frac{2}{L}\int_0^L &f(x)sin(\frac{n \pi x}{L})dx\\
    &\text{and}\\
    B_n = \frac{2}{L}\int_0^L &f(x)cos(\frac{n \pi x}{L})dx.
\end{align}

$\newline$
Orthogonality of sines and cosines (formulas):
\begin{align}
    \int_0^L sin(\frac{n\pi x}{L}) sin(\frac{m\pi x}{L}) = \begin{cases}
        0 \text{; if }n\neq m\\
        \frac{L}{2}\text{; if }n=m.
    \end{cases}\\
    \int_0^L cos(\frac{n\pi x}{L}) cos(\frac{m\pi x}{L}) = \begin{cases}
        0 \text{ ; if }n\neq m\\
        \frac{L}{2}\text{; if }n=m\neq 0\\
        L\text{ ; if }n=m =0
    \end{cases}
\end{align}










}


\end{document}