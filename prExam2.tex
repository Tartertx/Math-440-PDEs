\documentclass[12 pt]{article}        	%sets the font to 12 pt and says this is an article (as opposed to book or other documents)
\usepackage{amsfonts, amssymb}
\usepackage{amsmath}
\usepackage{graphicx}
\usepackage{esint}
\usepackage{float}
\usepackage{mathrsfs}

\usepackage[left=2cm,right=2.5cm,top=2cm,bottom=1.5cm]{geometry}

\pagestyle{myheadings}

\newcommand{\eqn}[0]{\begin{array}{rcl}}%begin an aligned equation - allows for aligning = or inequalities.  Always use with $$ $$
\newcommand{\eqnend}[0]{\end{array} }  	%end the aligned equation

\newcommand{\qed}[0]{$\square$}

% personalized commands
\newcommand{\inv}{^{-1}}
\newcommand{\bij}{\mathrlap{\hookrightarrow}\rightarrow}
\newcommand{\onto}{\twoheadrightarrow}
\newcommand{\z}{\mathbb Z}
\newcommand{\real}{\mathbb R}
\newcommand{\q}{\mathbb Q}
\newcommand{\complex}{\mathbb C}
\newcommand{\n}{\mathbb N}
\newcommand{\cont}{\lightning}
\newcommand{\osub}{\stackrel{\circ}{\subset}}
\newcommand{\U}{\mathscr{U}}
\newcommand{\B}{\mathscr{B}}

\usepackage{amsthm} % Recommended for theorem-like environments
\usepackage[usenames,dvipsnames]{xcolor}

\newtheorem{definition}{Definition}
\newtheorem{prop}{Proposition}
\newtheorem{ex}{Example}
\newtheorem{thm}{Theorem}
\newtheorem{lem}{Lemma}

\title{Math 440 Exam 2 Practice Test}
\begin{document}
\maketitle
\begin{center}
    Timothy Tarter

    James Madison University

    Department of Mathematics
\end{center}

\section{Describe / Explain:}
\subsection{Laplace's Equation (include well-posedness, uniqueness, and the mean value proposition).}
{\color{red} 
Laplace's equation states that the sum of the first partials are equal to zero, i.e.,
\begin{equation}
    0= \frac{\partial u }{\partial x} + \frac{\partial u }{\partial y} + \frac{\partial u }{\partial z}
\end{equation}

$\newline$
Well-posedness: there exists a unique solution which depends continuously on non-homogeneous data.

$\newline$
Uniqueness: solutions to laplace's equation are unique.

$\newline$
Mean value proposition: the temperature at the origin is the average of the temperature at the boundary.

}

\subsection{Linear Operators, provide an example of something which is a linear operator, and provide an example of something which is not a linear operator.}
{\color{red} A linear operator, $\U:V \to V$ where $V$ is a vector space, has the property that for any $\vec{x},\vec{y} \in V$ and any scalar $c$ in the field of $V$,
\begin{enumerate}
    \item $\U(\vec{x}+\vec{y}) = \U(\vec{x}) + \U(\vec{y})$
    \item $\U(c\vec{x}) = c\U(\vec{x})$.
\end{enumerate}

$\newline$
Example:
\begin{equation}
    L(u) = \frac{\partial}{\partial x} \left[ K_0 (x) \frac{\partial u}{\partial x} \right]
\end{equation}

$\newline$
Non-Example:
\begin{equation}
    L(u) = \frac{\partial}{\partial x} \left[ K_0 (u,x) \frac{\partial u}{\partial x} \right]
\end{equation}
where $K_0$ is a non-constant function.



}

\section{Prove Orthogonality of Sines.}
{\color{red} We want to show that

\begin{align}
    \int_0^L sin(\frac{n\pi x}{L}) sin(\frac{m\pi x}{L}) = \begin{cases}
        0 \text{; if }n\neq m\\
        \frac{L}{2}\text{; if }n=m.
    \end{cases}\\
    \int_0^L cos(\frac{n\pi x}{L}) cos(\frac{m\pi x}{L}) = \begin{cases}
        0 \text{ ; if }n\neq m\\
        \frac{L}{2}\text{; if }n=m\neq 0\\
        L\text{ ; if }n=m =0
    \end{cases}
\end{align}

$\newline$
Starting off with (4), if $n = m$, then 
\begin{equation}
    \int_0^L sin^2(\frac{n\pi x}{L})dx = \int_0^L \frac{1}{2}(1-cos(2\frac{n\pi x}{L}))dx
\end{equation}
\begin{align}
    &= \frac{1}{2}\left[x-\frac{L}{2n\pi}sin(\frac{2n\pi x}{L})\right]\bigg|_0^L\\
    &= \frac{1}{2}\left[ L - 0 \right] = \frac{L}{2}
\end{align}
as desired. Alternatively, if $n \neq m$, then
\begin{align}
    &\int_0^L sin(\frac{n\pi x}{L})sin(\frac{m\pi x}{L})dx\\
    &= \int_0^L \frac{1}{2}[cos(\frac{n\pi x}{L}-\frac{m\pi x}{L}) - cos(\frac{n\pi x}{L}+\frac{m\pi x}{L})]dx\\
    &= \frac{1}{2}\int_0^L cos(\frac{\pi x(n-m)}{L})- cos(\frac{\pi x(n+m)}{L})dx\\
    &= \frac{1}{2}[\frac{L}{\pi(n-m)}sin(\frac{\pi x (n-m)}{L})-\frac{L}{\pi(n-m)}sin(\frac{\pi x (n-m)}{L})]\bigg|_0^L\\
    &= 0 -0 = 0
\end{align}
as desired.

$\newline$
Moving to (5), if $n\neq m$, it works pretty much the same way as with sines. If $n = m \neq 0$, then 
\begin{align}
    &\int_0^L cos^2(\frac{n\pi x}{L})dx =\\
    &= \int_0^L \frac{1}{2}(1+sin(\frac{2n\pi x}{L}))dx\\
    &= \frac{1}{2}[x-\frac{L}{2n\pi}cos(\frac{2n \pi x}{L})]\bigg|_0^L \\
    &= \frac{1}{2}[L + (-1 + 1)]\\
    &= \frac{L}{2}
\end{align}
as desired.
Then finally, if $n=m=0$,
\begin{align}
    \int_0^L cos^2(0)dx = \int_0^L 1dx = L.
\end{align}




}

\section{Solve $\frac{\partial u}{\partial t} = k \frac{\partial^2 u}{\partial x^2}$ with certain BC's and IC's, addressing $\lambda >0, = 0, <0$.}
\subsection{$u(0,t) = u(L,t) = 0$}
{\color{red} Gives sine solutions.}

\subsection{$\frac{\partial u}{\partial x}(0,t) = 0 = \frac{\partial u}{\partial x}(L,t)$}
{\color{red} Gives cosine solutions.}

\subsection{$u(0,t) = u(L,t) = 0$ and $\frac{\partial u}{\partial x}(0,t) = 0 = \frac{\partial u}{\partial x}(L,t)$}
{\color{red} Gives sine and cosine solutions.}

\section{Solve $\frac{\partial u}{\partial t} = k \frac{\partial^2 u}{\partial x^2}$ at equilibrium in polar coordinates over the interval $(-L, L)$.}
{\color{red}     

We begin with the one-dimensional heat equation
\[
u_t = \kappa u_{xx}, \qquad x \in [-L, L],
\]
and bend the wire so that its endpoints are in perfect thermal contact:
\[
u(-L, t) = u(L, t), \qquad u_x(-L, t) = u_x(L, t).
\]
These are the \emph{periodic boundary conditions}, corresponding physically to a wire bent into a circle of circumference \(2L\).

$\newline$
At equilibrium, the temperature no longer depends on time, so
\[
u_t = 0 \quad \Longrightarrow \quad u_{xx} = 0.
\]
Integrating twice gives
\[
u_{\text{eq}}(x) = A x + B.
\]
The periodicity condition \(u_{\text{eq}}(-L) = u_{\text{eq}}(L)\) yields
\[
A(-L) + B = A(L) + B \quad \Longrightarrow \quad A = 0.
\]
Hence
\[
\boxed{u_{\text{eq}}(x) = B = \text{constant}.}
\]

$\newline$
To determine \(B\), note that the total heat (or average temperature) is conserved for periodic boundary conditions:
\[
\frac{d}{dt}\int_{-L}^{L} u(x,t)\,dx
= \kappa\,[u_x(x,t)]_{x=-L}^{x=L}
= \kappa\,(u_x(L,t) - u_x(-L,t)) = 0.
\]
Thus, the equilibrium constant equals the initial mean temperature:
\[
\boxed{
u_{\text{eq}}(x) = \frac{1}{2L}\int_{-L}^{L} u_0(x)\,dx.}
\]

\[
\text{Therefore, the bent wire reaches a uniform equilibrium temperature equal to its initial average.}
\]

}

\section{Solve Laplace's Equation on a rectangle, with LHS, Top, and Bottom $=0$, RHS $= g_2(y)$.}

{\color{red}  We want to start with Laplace's 2D equation,
\begin{equation}
    0= k[\frac{\partial^2 u}{\partial x} + \frac{\partial^2 u}{\partial y}].
\end{equation}
If $u(x,t) = X(x)Y(y)$, then
\begin{equation}
    0 = X''Y+XY''
\end{equation}
implies that
\begin{equation}
    \frac{X''}{X}=\lambda = \frac{-Y''}{Y}.
\end{equation}
Thus,
\begin{equation}
    X''-\lambda X = 0
\end{equation}
and
\begin{equation}
    Y''+\lambda Y=0
\end{equation}
are our system of ODE's. Now we want to specify our boundary conditions. 

$\newline$
This results in the following BCs:
\begin{itemize}
    \item $u(0,y) = 0$
    \item $u(L,y) = g_2(y)$
    \item $u(x,0) = 0$
    \item $u(x,H)=0$.
\end{itemize}
Accordingly, 
\begin{equation}
    Y(y) = \sum_{n=1}^\infty A_n sin(\frac{n\pi y}{H}).
\end{equation}
Then,
\begin{equation}
    X''-\lambda X = 0
\end{equation}
implies that 
\begin{equation}
    X(x) = Asinh(\sqrt{\lambda}x) + B cosh(\sqrt{\lambda} x).
\end{equation}
Since 
\begin{equation}
    X(0) = B = 0,
\end{equation}
then 
\begin{equation}
    X(x) = Asinh(\frac{n\pi x}{H}).
\end{equation}
But,
\begin{align}
    X(L) = g_2(y) = Asinh(\frac{n\pi L}{H}) \\
    A = \frac{g_2(y)}{sinh(\frac{n\pi L}{H})}.
\end{align}
So 
\begin{equation}
    u(x,y) = \sum_{n=1}^\infty A_n sin(\frac{n\pi y}{H})\frac{g_2(y)sinh(\frac{n\pi x}{H})}{sinh(\frac{n\pi L}{H})}
\end{equation}
where
\begin{equation}
    A_n = \frac{2}{H}\int_0^Hg_2(y)sin(\frac{n\pi y}{H})dy.
\end{equation}







}




\section{Extra Resources Which You Gotta Know or You Might Be Cooked:}
{\color{blue} 
Trig Identities: 
\begin{align}
    sin(a+b) = sin(a)cos(b)+cos(a)sin(b)\\
    cos(a+b) = cos(a)cos(b) - sin(a)sin(b)\\
    sin(a-b) = sin(a)cos(b)-cos(a)sin(b)\\
    cos(a-b) = cos(a)cos(b) + sin(a)sin(b)
\end{align}

$\newline$
If BC's are
\begin{itemize}
    \item $u(0,t) = 0 = u(L,t)$ then 
    \begin{equation}
        u(x,t) = \sum_{n=1}^\infty A_n e^{-kt(\frac{n\pi}{L})^2}sin(\frac{n\pi x}{L})
    \end{equation}
    \item $\frac{\partial u}{\partial x}(0,t) = 0 = \frac{\partial u}{\partial x}(L,t)$ then 
    \begin{equation}
        u(x,t) = \sum_{n=0}^\infty B_n e^{-kt(\frac{n\pi}{L})^2}cos(\frac{n\pi x}{L})
    \end{equation}
\end{itemize}
where 
\begin{align}
    A_n = \frac{2}{L}\int_0^L &f(x)sin(\frac{n \pi x}{L})dx\\
    &\text{and}\\
    B_n = \frac{2}{L}\int_0^L &f(x)cos(\frac{n \pi x}{L})dx.
\end{align}

$\newline$
Orthogonality of sines and cosines (formulas):
\begin{align}
    \int_0^L sin(\frac{n\pi x}{L}) sin(\frac{m\pi x}{L}) = \begin{cases}
        0 \text{; if }n\neq m\\
        \frac{L}{2}\text{; if }n=m.
    \end{cases}\\
    \int_0^L cos(\frac{n\pi x}{L}) cos(\frac{m\pi x}{L}) = \begin{cases}
        0 \text{ ; if }n\neq m\\
        \frac{L}{2}\text{; if }n=m\neq 0\\
        L\text{ ; if }n=m =0
    \end{cases}
\end{align}










}


\end{document}